\documentclass[twoside,12pt,openright]{report}
\usepackage[utf8]{inputenc}
\usepackage[italian]{babel}
\usepackage[a4paper,top=3cm,bottom=3cm,left=3cm,right=3cm,marginparwidth=1.5cm,bindingoffset=15mm,heightrounded]{geometry}
\usepackage{csquotes}
\usepackage{quoting}
\quotingsetup{font=small}

% Times new Roman font e interlinea a 1.5 (mandatorie Mercatorum) 
\usepackage{times}
\linespread{1.5}

% Per creare box colorati
\usepackage{tcolorbox} 

\usepackage{mdframed}

%Stile per la vignetta
\tcbset{
    myvignette/.style={
        colback=gray!20,    % Sfondo grigio chiaro
        colframe=black,     % Bordo nero
        coltitle=white,     % Titolo bianco
        fonttitle=\bfseries,% Titolo in grassetto
        boxrule=1pt,        % Spessore del bordo
        rounded corners,    % Angoli arrotondati
        width=\textwidth,   % Larghezza pari alla pagina
        sharp corners=north,% Angolo superiore non arrotondato
        title={Caso Particolare} % Titolo di default
    }
}

\usepackage{amssymb}
\usepackage{amsmath}
\usepackage{graphicx}
\usepackage[colorlinks=true, allcolors=black]{hyperref}

%Per inserire codice
\usepackage{xcolor}% Per la colorazione del codice
\usepackage{colortbl} 
\usepackage{array}
\usepackage{listings}

\lstset{
    language=Python,
    basicstyle=\ttfamily\scriptsize,
    %baselinestretch=0.9,
    keywordstyle=\color{blue},
    commentstyle=\color{green!50!black},
    stringstyle=\color{red},
    numbers=left,
    numberstyle=\tiny\color{gray},
    stepnumber=1,
    numbersep=5pt,
    backgroundcolor=\color{gray!10},
    showspaces=false,
    showstringspaces=false,
    showtabs=false,
    frame=single,
    rulecolor=\color{black},
    tabsize=4,
    captionpos=b,
    breaklines=true,
    breakatwhitespace=false,
    escapeinside={(*@}{@*)},
    xleftmargin=17pt,
    framexleftmargin=17pt,
    framexrightmargin=5pt,
    framexbottommargin=4pt
}

\lstdefinestyle{MatlabStyle}{% Se si usa un altro linguaggio, modificare "language"
    language=mathlab,
    basicstyle=\ttfamily\footnotesize,
    keywordstyle=\color{blue},
    commentstyle=\color{green!50!black},
    stringstyle=\color{red},
    backgroundcolor=\color{gray!10},
    frame=single,
    rulecolor=\color{black},
    breaklines=true,
    numbers=left,
    numberstyle=\tiny, 
    stepnumber=1,
    numbersep=5pt
}

\usepackage{booktabs}
\usepackage{tikz}

% TikZ libraries
\usetikzlibrary{positioning, arrows.meta, shapes.geometric}

% TikZ styles
\tikzstyle{vertex}=[circle, draw, thick, minimum size=15pt, inner sep=1pt]
\tikzstyle{edge label}=[midway, fill=white, inner sep=1pt, font=\footnotesize]
\tikzstyle{directed edge}=[-{Latex[length=2mm]}, thick]
\tikzstyle{undirected edge}=[thick]

\usepackage{algorithm}% For algorithm environment
\usepackage{algpseudocode} % For pseudocode layout
% Define argmin operator if not available
\DeclareMathOperator*{\argmin}{arg\,min}

\usepackage[style=apa, backend=biber]{biblatex}
%\usepackage[style=numeric,sorting=none]{biblatex}
\addbibresource{bibliografia.bib}

%----------------------------MAIN-----------------------------------

\begin{document}

\newgeometry{top=3cm,bottom=3cm,left=3cm,right=3cm,heightrounded}
\begin{titlepage}
    \begin{figure}[!htb]
        \centering
         \includegraphics[keepaspectratio=true,scale=3.5]{Immagini/Mercatorum logo.jpg} % Placeholder
        %\rule{5cm}{3cm} % Placeholder for the image
    \end{figure}

    \begin{center}
        \vspace{5mm}
        \large{\textbf{FACOLTÀ DI SCIENZE TECNOLOGICHE E DELL'INNOVAZIONE}}
        \vspace{1.5mm}

        \noindent\rule{225pt}{0.4pt}

        \vspace{5mm}

        \large{\textbf{CORSO DI LAUREA IN ING. INFORMATICA}}

        \vspace{10mm}

        \Large{\textbf{\emph{Prova Finale in}}}

        \large{\textbf{PROGRAMMAZIONE}}

        \vspace{1mm}

        \large{Architetture Software per gli Investimenti nell’Era dell’AI}

    \end{center}

    \vspace{15mm}

    \begin{minipage}[t]{0.33\textwidth} % Relatore minipage (kept for context)
        \centering % Center the content within this minipage
        {\large RELATORE\\[0.5em]
         \normalsize Chiar.mo\\ % Removed centering here, as \centering above handles it
         Bellone Mauro}
    \end{minipage}
    \hfill % Pushes the minipages apart
    \begin{minipage}[t]{0.35\textwidth} % <-- REMOVED \raggedleft
        \centering % <--- ADDED \centering switch here
        {\large CANDIDATO\\[0.5em] % Braces keep \large local
         \large Andrea Poletti\\[0.5em]
         \normalsize MATR. 0082101302} % Braces keep \normalsize local
    \end{minipage}

    \vspace{20mm}

    \centering{\normalsize{Anno Accademico 2024/2025}}

\end{titlepage}
\restoregeometry


\newpage
\null
\newpage

 
\vspace{30cm}


\begin{flushright}
\emph{Dedicato a\\
Mamma e\\
Papà}
\end{flushright}

\tableofcontents %Indice

\chapter*{Abstract} 
\addcontentsline{toc}{chapter}{Abstract}

L'automazione finanziaria, potenziata e supportata dalla diffusione delle intelligenze artificiali (AI) degli ultimi anni, sta ridefinendo il panorama degli investimenti retail in Italia e a livello globale.

Nel 2025 il mercato dei robo-advisor (R-A) ha raggiunto un valore stimato di \$14 miliardi, con una crescita annua (dati CAGR) del 30\%, trainata da piattaforme bancarie come MoneyFarm, che offrono soluzioni accessibili con commissioni basse (0.25-0.5\%).

Parallelamente, il trading online, ed in modo più specifico il trading con algoritmi reso popolare da piattaforme come MetaTrader5, consente agli investitori privati di sviluppare strategie personalizzate, con vari gradi di rischio e potenziale rendimento.

In Italia il contesto normativo (MiFID II) distingue il regime amministrato con responsabilità fiscale gestita dalla banca, dal regime dichiarativo, che invece richiede competenze tecniche e responsabilità diretta.

Questo dualismo solleva una domanda cruciale: quale approccio è più adatto all’investitore medio di oggi?

\chapter{Introduzione}
\section{Contesto}

Per rispondere a questa domanda, è prima necessario comprendere il contesto storico.

La crisi finanziaria associata alla pandemia da COVID-19 ha rappresentato un catalizzatore significativo per l'evoluzione della cultura finanziaria tra i giovani di tutto il mondo, evidenziando vulnerabilità pre-esistenti e stimolando una maggiore consapevolezza: inflazione alle stelle, stipendi congelati e cultura sociale del paragone fanno sorgere nuove necessità, seppure i dati indichino l'assenza di un incremento netto dei livelli di alfabetizzazione finanziaria - punteggio italiano OCSE PISA di 484 nel 2022, al di sotto della media OCSE (498), con il 18\% dei 15enni sotto il livello base.

In Italia, quasi il 40\% dei giovani (18-34 anni) è incapace di fronteggiare imprevisti di 2.000€. Inoltre, il 54\% riferisce di percepire la crisi come una minaccia grave. Tutto ciò accentua le disuguaglianze di genere e territoriali (ad esempio donne e residenti al Sud).

Questa diffusa paura legata alla difficoltà intrinseca di capire il complesso mondo della borsa genera un mix efficace: il mondo dei social media viene letteralmente invaso dalle figure dei "guru finanziari", le piattaforme di trading spopolano sugli app-store e l'ideologia generale tende alla ricerca della cosiddetta work-life balance e del passive income.

Anche le Banche presentano la loro proposta per rispondere ad un pubblico digitalizzato, con pochi fondi di partenza e che non può quindi sostenere commissioni alte: nascono i Robo-Advisor, strumenti automatizzati per la gestione degli investimenti. 
Il cliente retail può iniziare con cifre modeste (solitamente da 500€ a 5.000€), sostenendo commissioni basse (in media fra 0.25-0.5\%), a cui associa un profilo di rischio individuato tramite questionario, e a cui può eventualmente aggiungere mensilmente una cifra.

\section{Ipotesi}

Questa tesi vuole analizzare l’impatto di questi servizi, confrontando i rendimenti di un R-A in regime amministrato (verrà preso in esame in particolare MoneyFarm, uno fra i più diffusi e redditizi, e soprattutto che rende disponibili pubblicamente i propri dati) con un algoritmo di trading “semplice” sviluppato ad hoc (in questo caso verrà utilizzata la strategia moving-average crossover) che opera in loop su una piattaforma gestita da broker.

L’ipotesi di ricerca valuta se, ad oggi, sia possibile creare ed utilizzare un algoritmo personalizzato che possa imitare il metodo d’investimento di un R-A, superarne i rendimenti a fronte di un maggiore involvement da parte dell’utente, con un livello di rischio simile ai profili forniti dalle Banche.
Questo per tutelare l’utente con literacy bassa ed insieme fornire la tranquillità necessaria rimuovendo la componente dell’emotività umana dalla strategia.
Quanto può influire, se non meglio costare, un minimo livello di cultura finanziaria, sorprendentemente assente fra i giovani italiani, sui risparmi?


\section{Metodologia}

La metodologia prevede la creazione di un algoritmo per l’operatività, che raccoglie dati da fonti ufficiali e li utilizza per implementare una specifica strategia (a valori configurabili).
Verranno effettuate simulazioni empiriche tramite Python su dataset storici, considerando fattori reali quali commissioni e tasse italiane. 
I risultati verranno confrontati con le performance dichiarate da MoneyFarm, valutando metriche come rendimento (patrimonio finale), volatilità (a simboleggiare il rischio), drawdown (massima perdita), ed indice di Sharpe (indice rischio/rendimento).

La tesi presenta il diagramma UML dell’architettura software utilizzata, correlato a specifiche descrizioni dei vari blocchi necessari all'espletamento delle funzioni.

Il lavoro contribuisce al dibattito sull'automazione finanziaria, offrendo un'analisi tecnica per investitori e sviluppatori.

È importante evidenziare che, attraverso i materiali qui presenti, NON vengono forniti consigli finanziari né consulenza finanziaria. Le opinioni tecniche condivise hanno il solo scopo di formazione ed informazione. La decisione finale riguardante ogni investimento è una scelta personale e individuale del lettore.


\chapter{Stato dell'Arte dei Robo-Advisor}

In questo capitolo verrà esplorato brevemente il mondo dei R-A, per individuare metodi e strumenti da poter utilizzare come paragone per la creazione dell'algoritmo di trading.

\section{Panoramica Software Robo-Advisor}

%Cosa sono, distinzioni
I R-A rappresentano una delle innovazioni più significative nell’ambito dell'automazione finanziaria degli ultimi anni: essi sono in grado di combinare intelligenza artificiale e algoritmi di ottimizzazione per gestire portafogli a livello retail, con bassi costi di partenza e discrete opzioni di personalizzazione.

%Diffusione storica
In Italia, piattaforme come MoneyFarm, Tinaba (Banca Profilo) e servizi bancari di Intesa SanPaolo e Unicredit dominano il mercato, offrendo soluzioni in regime amministrato che semplificano la gestione fiscale per gli investitori. 
La normativa distingue il regime amministrato da quello dichiarativo per la gestione della fiscalità: la Banca si preoccupa di versare i contributi per il cliente, che visualizza direttamente le plus/minus valenze al netto della tassazione.

%Come funzionano
Il cliente privato non deve quindi fare altro che depositare il denaro sul conto, rispondere a qualche domanda per individuare il profilo di rischio tollerato, ed osservare saltuariamente il grafico del rendimento, eliminando completamente l'emotività umana dall'equazione.
Questa breve descrizione lascia quindi intuire il motivo della diffusione di questi strumenti negli ultimi anni: comodità di utilizzo ed affidabilità associata al nome della specifica istituzione.

Il funzionamento si basa su architetture software che integrano tre componenti principali: raccolta dati utente tramite questionari digitali, ottimizzazione del portafoglio (es. modello di Markowitz per asset allocation) e rebalancing automatico utilizzando algoritmi ed AI. 
Nel 2025, il mercato italiano dei robo-advisor conta Assets Under Management (AUM) di circa \$60 miliardi (considerando anche sistemi ibridi), con una crescita annua del 25-30\%, secondo report CONSOB.
La semplicità operativa e i costi bassi rendono i R-A accessibili, ma limitano la personalizzazione rispetto al trading algoritmico: il cliente non ha la possibilità di scegliere come e in cosa investire.

%Grafici ufficiali diffusione RA
Per visualizzare l’adozione dei R-A, si riporta un grafico (figura 2.1) di trend degli AUM in Italia (2018-2025) convertiti in dollari (USD), dati basati su report CONSOB e CAGR.
La crescita esponenziale riflette l’aumento della fiducia degli investitori retail, con un picco 2020-2021 dovuto alla digitalizzazione accelerata dalla pandemia.

\begin{figure}[htbp]
    \centering
    \includegraphics[width=\textwidth]{Immagini/Trend AUM in Italia 2018-2025 (USD).png}
    \caption{AUM Italia, fonti CAGR e CONSOB}
    \label{fig:aum italia}
\end{figure}

\section{Composizione del Portafogli}

%Perchè MoneyFarm
Per poter creare un sistema che simula gli investimenti di un R-A, è importante capire come esso funziona, come vengano composti i suoi portafogli e quali siano i relativi costi di gestione.
E' dunque il caso di scegliere uno dei servizi R-A come base di partenza su cui poi verrà costruito l'algoritmo.
Tra i vari disponibili in Italia nel 2025, fra cui Gimme5, Tinaba, Euclidea, Revolut, Scalable Capital ed eToro;\textbf{MoneyFarm} si distingue come una delle opzioni più consolidate e premiate.

È stato infatti votato per il quinto anno consecutivo come il migliore servizio di consulenza finanziaria in Italia, posizionandosi come il R-A numero uno nel Paese.
Nel caso di MoneyFarm i portafogli sono gestiti come soluzioni diversificate, costruite principalmente su ETF (Exchange Traded Funds, divenuti popolari negli ultimi anni: sono pacchetti già diversificati attorno ad un tema centrale, per esempio energie rinnovabili, tech, mercati emergenti, ecc) per bilanciare rischio e rendimento in base alla profilazione dell'investitore. 

%Profili e composizione portafogli
I portafogli di MoneyFarm sono diversificati su asset class globali. L'importante esposizione al settore tech (da cui seguirà la scelta del simbolo su cui l'algoritmo investirà, descritta nel Capitolo 3) deriva indirettamente dalla porzione azionaria, che include indici come MSCI World o S\&P 500, dove il tech rappresenta circa il 25-40\%.

I profili vanno da 1 (più conservativo) a 7 (più aggressivo), con allocazioni che variano in percentuale tra obbligazioni per la stabilità e azioni per la crescita.
Per semplicità, in questa ricerca verranno considerati solamente i profili 1 (20/80 azionario), 4 (60/40 azionario) e 7 (80/20 azionario), da utilizzare per i confronti finali nel Capitolo 5.

\section{Rendimenti e Metriche Dichiarati}
%Rendimenti e metriche per confronto
I rendimenti medi dei R-A italiani si attestano tra il 4\% e il 6\% annuo; MoneyFarm nello specifico dichiara rendimenti molto competitivi rispetto alla concorrenza, con uno Sharpe ratio medio di 0.8-1.0 (l'indice di Sharpe rappresenta il rapporto fra potenziale guadagno e volatilità. Più è alto, più l'investimento è indicativamente buono).

\begin{figure}[htbp]
    \centering
    \includegraphics[width=\textwidth]{Immagini/MoneyFarm Rendimento Grafico.png}
    \caption{MoneyFarm - Rendimenti dal 2018}
    \label{fig:moneyfarm grafico}
\end{figure}

Il grafico in figura 2.2 mostra i rendimenti (lordi) dichiarati da MoneyFarm per il profilo 7 (aggressivo) dal 2018, estratti direttamente dal sito web ufficiale.
I costi operativi totali (inclusi gestione e commissioni) verranno fissati al 1.28\%, cosi come al 26\% la tassazione italiana sul capital gain.
Questi dati forniranno da base (benchmark) per il confronto con l’algoritmo di trading sviluppato per la ricerca, esplorato in dettaglio nel capitolo 5.


\chapter{Sistemi Software per Trading Algoritmico}
%Intro, da ordinare bene
Il trading algoritmico consente di automatizzare strategie di investimento tramite un sistema software personalizzato.
In questo capitolo verrà descritto il metodo di sviluppo del programma, la strategia scelta per i test, gli strumenti utilizzati e il diagramma delle attività del main loop.

\section{Premessa: Benchmark, Periodo e Strategia}
%Criteri della tesi
Prima di proseguire, è importante fare un'overview delle scelte fatte in fase di concepimento: nel capitolo 2 è stata analizzata la struttura del R-A di MoneyFarm, i suoi KPI e la composizione dei portafogli. 
I criteri da rispettare perchè il test sia il più attendibile possibile sono quindi chiari: la struttura dell'algoritmo dovrà essere semplice, il portafoglio dovrà rispecchiare quanto più possibile quello utilizzato dal R-A, e il periodo di test dovrà incorporare momenti di crescita così come di crisi.
Questo approccio, volto all'accessibilità, richiede competenze informatiche moderate e risorse computazionali minime.

%Perchè Nasdaq
La configurazione dell'algoritmo accetterà una lista di simboli (strumenti d'investimento).
Verrà utilizzato inizialmente solo l'indice Nasdaq-100, simbolo del periodo AI-driven di questi anni.
Come discusso nel Capitolo 2, i Robo-Advisor tendono ad investire prevalentemente in ETF tech, riflettendo l'evoluzione del mercato verso settori innovativi.
Nel 2025, si è assistito ad un significativo shift dei mercati verso l'AI, con investimenti a livelli mai visti prima in questo ambito.

Basti pensare che, rispetto a tutte le aziende presenti nell'indice Nasdaq-100, circa il 49\% del capitale è concentrato solo fra le prime cinque: Nvidia, Apple, Microsoft, Amazon e Alphabet (Google). 
Di fatto, la gran parte degli ETF disponibili, composti da S\&P-500 o Nasdaq-100, sono partecipati per oltre il 40-50\% da queste specifiche aziende, tutte fortemente legate all'AI e alla tecnologia. 
È quindi estremamente difficile costruire un portafoglio senza partecipare attivamente all'investimento nell'AI, sia per piccoli che per grandi investitori.

La scelta del simbolo, come già accennato, è pertanto ricaduta sul Nasdaq-100: un indice che rappresenta l'innovazione e l'economia moderna, focalizzandosi sulle aziende tecnologiche e ad alto potenziale di crescita.
Lanciato nel 1985 dal Nasdaq Stock Market (fondato nel 1971, è stato il \textit{primo scambio computerizzato al mondo}), il Nasdaq-100 è composto da 100 delle più grandi aziende non-finanziarie quotate sull'indice, selezionate in base alla capitalizzazione di mercato e ponderate di conseguenza.
I settori principali includono tecnologia (circa il 60\%), consumer discretionary (18\%), healthcare (6\%) e telecomunicazioni (5\%), con un'enfasi su giganti innovativi come quelli prima menzionati. 
Oggi, nel 2025, riveste un'importanza cruciale in borsa, con oltre 200 prodotti di tracking e più di 600 miliardi di dollari in AUM, simboleggiando la dominanza dell'AI e della digitalizzazione nei mercati globali.
Questa scelta è significativa per il backtest poiché l'indice cattura trend volatili e innovativi, offrendo un benchmark realistico per strategie algoritmiche in contesti moderni.

%Perchè dal 2018
Il backtesting dell’algoritmo verrà condotto su dati storici dal 2018 ad oggi, per coprire fasi di mercato volatili (es. post-COVID 2020) e stabili (2023-2025), offrendo un contesto robusto per valutare la strategia.
Infine, verrà eseguito un test su simboli multipli per verificare la variazione delle performance una volta introdotto il concetto di diversificazione.

%Strategia
Per quanto riguarda la strategia, verranno utilizzate due medie mobili semplici (SMA): una a breve e una a lungo termine, che identificano il trend del prezzo di chiusura del simbolo/i selezionato/i. 
Un segnale si genererà all'intersecarsi delle due: un segnale di acquisto quando la media breve supera la linea di trend a lungo termine, mentre un segnale di vendita al crossover opposto. 
In figura 3.1 si osserva la strategia evidenziata: la linea verde rappresenta la media mobile 50, mentre la linea rossa la 200. All'intersecarsi, il programma rileverà il crossover e calcolerà l'eventuale operazione da eseguire.

%3.1 MA Crossover Strategy
\begin{figure}[htbp]
    \centering
    \includegraphics[width=\textwidth]{Immagini/MA-Crossover .jpg}
    \caption{Strategia MA Crossover}
    \label{fig:UML MA Crossover}
\end{figure}

Questo metodo è molto conservativo: si può facilmente notare come il prezzo (linea blu) \textbf{inizi la discesa ben prima} che le due medie si intersechino.
Questo accade poichè la media mobile è un valore ottenuto dai dati del passato e non del presente. 
Il segnale quindi risulta in ritardo rispetto al mercato, lasciando sul tavolo potenzialmente molto guadagno. 
Unendo questa strategia all'indicatore RSI (Relative Strenght Index), che confermerà la forza del movimento, si eviteranno falsi segnali: se il crossover è stato rilevato ma il movimento è ormai rallentato (soglie di oversold - asset molto venduto, e di overbought - acquisto molto popolare), l'algoritmo eviterà di aprire posizioni troppo rischiose.
Il metodo non verrà volutamente reso più complesso.

%3.2 UML Activity
\begin{figure}[htbp]
    \centering
    \includegraphics[width=\textwidth]{Immagini/UML Activity Simplified.png}
    \caption{Diagramma UML delle Attività}
    \label{fig:UML Activity}
\end{figure}

\section{Diagramma delle Attività}
Si riporta in figura 3.2 il diagramma UML semplificato delle attività del programma.
L’architettura software seguirà questo flusso: configurazione dei profili; acquisizione dati di mercato; calcolo della strategia e generazione dei segnali di trading; invio segnali ad MT5 e simulazione delle posizioni per il backtesting; stampa dei grafici.

Viene evidenziata nel grafico la modularità del sistema, implementato in modo da accettare in ingresso più simboli così da poter diversificare a piacimento il portafoglio, e più livelli di rischio con i relativi configuration data.
Il programma è stato scritto in Python, sfruttando varie librerie: yinance per l'acquisizione dati, pandas e numpy per l’elaborazione, ta-lib e matplotlib per la stampa dei grafici e MetaTrader5 per l'interfaccia con la piattaforma di trading vera e propria.


\section{Parametri di Configurazione}
%Config
I parametri più importanti da discutere, da cui dipenderà la mitigazione del rischio dei singoli profili, sono visualizzati nell'estratto di codice da "account\_data.py" (listing 3.1), un file dedicato ai valori di configurazione per personalizzare l'algoritmo.
E' stata prevista in fase progettuale la possibilità di distinguere vari profili di rischio, cosi da eseguire in parallelo i test e poter confrontare i risultati della stessa strategia con diverse tolleranze.
In apertura si osservano gli indicatori di questi "limiti": SHORT\_MA e LONG\_MA sono i periodi (in giorni) con cui verrà calcolata la media mobile semplice.
I valori di RSI sono il periodo (in giorni) fra cui calcolare l'indicatore, ed i relativi valori di soglia (OVERSOLD, OVERBOUGHT) per decidere se il segnale è falso o attendibile.

%Metodo di stop loss
Perchè un algoritmo di trading sia veramente indipendente dall'input dell'operatore, oltre a capire quando aprire le posizioni dovrà anche essere in grado di chiuderle autonomamente. Vengono introdotti i valori di Trailing Stop Loss (TRAIL\_PERCENT).
Una volta generato un segnale ed aperta una posizione, l'algoritmo fisserà una percentuale di perdita accettabile.
Questa sarà la rete di sicurezza che non permetterà di scendere sotto una determinata soglia di perdita durante la singola operazione, per conservare il capitale.
Non è stato intenzionalmente fissato un valore di Take Profit (percentuale a cui la posizione in guadagno si chiude automaticamente), per evitare perdite in condizioni di lateralizzazione, ovvero quando il mercato non sale nè scende.
Verrà utilizzata una safety net "mobile" che seguirà fedelmente il prezzo della posizione nel caso dovesse crescere, cosi da lasciar correre in salita il lotto e fermarlo solo nell'eventualità che rintracci oltre una certa percentuale di perdita.
Anche il valore di Trailing Stop Loss verrà personalizzato in accordo con il tipo di profilo.

\begin{lstlisting}[
    caption={Parametri di Configurazione},
    label={lst:config},
    float=htbp
]
#Strategy parameters with profiles:
PROFILES        = ["high", "medium", "low", "pac"]
SHORT_MA        = [10,  20, 50, None]   #days
LONG_MA         = [50,  100, 120, None]
RSI_PERIOD      = [9,   14,  14, None]  #days    
RSI_OVERBOUGHT  = [40,  50,  65, None]
RSI_OVERSOLD    = [60,  50,  35, None]
TRAIL_PERCENT   = [0.03, 0.02,  0.015]  # 3%, 2%, 1,5%

#Account parameters
SYMBOLS         = ["^NDX", "^SPX", "^GDAXI"]
INITIAL_DEPOSIT = 5000 #USD
COMMISSION      = 0.0005 #spread commission on trade

#PAC parameters
PAC_MONTHLY_INVESTMENT = 100 #USD
\end{lstlisting}

Quindi riassumendo: il profilo HIGH sarà il più speculativo, la finestra di giorni per il calcolo del trend breve e lungo è più stretta, in modo da rilevare più punti di crossover e generare potenzialmente più operazioni; inoltre la soglia di RSI è rilassata, per abbassare le difese.
Rendendo invece il periodo SMA più ampio, il potenziale di operatività viene abbassato, creando così il profilo LOW.
Il profilo MEDIUM è una via di mezzo tra i due precedenti.
Come visibile nell'estratto di codice, è stato aggiunto anche un profilo PAC. Questo opera seguendo una strategia ancora più semplice, venendo gestito come un piano di accumulo del capitale (PAC). Eseguirà mensilmente operazioni di acquisto di posizioni long per la somma di 100\$ (scelta arbitraria per dividere gli acquisti in 50 mesi consecutivi), fino all'investimento dell'intero capitale di partenza (5.000\$).

\section{Acquisizione Dati ed Elaborazione Strategia}
%Data aquisition
Il file "import\_data.py" è dedicato all'acquisizione dei dati storici da una solida fonte nel settore: per questa ricerca è stato selezionato Yahoo Finance.
Il programma verifica l'esistenza di un file .csv con i dati relativi allo specifico simbolo selezionato, lo aggiorna (o crea) alla data odierna come visibile nel listing 3.2.
Vengono memorizzati valori quali prezzo di apertura e chiusura, punti high e low del giorno e volume di movimento.

\begin{lstlisting}[
    caption={Tabella dei Dati Storici, per simbolo},
    label={lst:config},
    float=htbp
]
#Date,      Open,       High,       Low,        Close,      Volume
2025-12-12, 25531.550,  25605.880,  25104.679,  25196.730,  8724070000
2025-12-15, 25352.869,  25377.619,  25022.810,  25067.269,  8649240000
2025-12-16, 24991.490,  25188.759,  24922.939,  25132.939,  7759960000
2025-12-17, 25167.859,  25193.410,  24647.609,  24647.609,  8616140000
2025-12-18, 25031.490,  25164.179,  24921.449,  25019.369,  7977920000
2025-12-19, 25147.300,  25354.820,  25134.259,  25346.179,  12874560000
\end{lstlisting}

Successivamente il file "signals\_generation.py" elabora i file .csv con i dati storici, ed in accordo con la strategia crea segnali dove lo ritiene opportuno, utilizzando le soglie previste dallo specifico profilo.
Un esempio è visibile in listing 3.3.
Nel caso un segnale venga generato, il dato viene memorizzato in un file .csv dedicato, utilizzato in seguito per il backtesting.

\begin{lstlisting}[
    caption={Esempio di Generazione di segnale Buy},
    label={lst:signal generation},
    float=htbp
]
#Buy signal
    df.loc[(df['SMA_short'] > df['SMA_long']) &
        (df['Prev_short'] <= df['Prev_long']) &
        (df['RSI'] <= oversold), 'Signal'] = 1
\end{lstlisting}

\section{Backtesting della Strategia ed Integrazione MT5}

%Come sono stati condotti i test
L’algoritmo è stato testato con un capitale simulato di \$5.000, assumendo commissioni di trading tipiche di broker italiani (0.05\% fisso per operazione, arrotondato in eccesso).
Viene sfruttato il file .csv prima menzionato con memorizzati i segnali generati per ogni profilo, e i risultati dei test (guadagni/perdite ed equity finale) vengono a loro volta salvati in un file distinto per generare poi i grafici di confronto.

%Integrazione con MT5
Il file "metatrader\_integration.py" è stato dedicato all'invio dei segnali alla vera e propria piattaforma di trading, i cui dati di login sono memorizzati in "account\_data.py".
E' stato scelto MetaTrader5 per l'ottima integrazione con Python, ma sostituendo questo file con uno equivalente per un'altra piattaforma il programma è in grado di funzionare allo stesso modo.
I risultati da cui verranno formate le conclusioni della tesi sono quelli del backtesting e non del live account su MT5 poichè, come visibile nel capitolo 4, la strategia MACrossover spesso non genera segnali per parecchi mesi.

\section{Stampa dei Grafici delle Performance}

Infine il file "print\_graphs.py" è dedicato alla visualizzazione dei risultati ottenuti dal backtesting, per sviluppare i successivi capitoli 4 e 5.
Sono stati scelti diversi tipi di grafico in base alla complessità dei dati da visualizzare, fra cui istogrammi per confrontare le performance e semplici tabelle per riassumere le percentuali di gain/loss.
Sono state utilizzate colorazioni distinte per rappresentare i diversi profili di rischio attraverso tutti i grafici: verde per LOW, arancione per MEDIUM, rosso per HIGH e blu per PAC.



\chapter{Risultati e Confronto}

In questo capitolo vengono analizzati i risultati ottenuti tramite backtesting della strategia sui dati storici forniti da Yahoo Finance.
La simulazione confronta le performance della strategia algoritmica creata ad-hoc con i valori di performance indicati sul sito web ufficiale di Moneyfarm, nel periodo dal 2018 ad oggi.
I grafici contengono i risultati per i tre diversi profili di rischio (basso, medio e alto), oltre a una strategia di investimento periodico (PAC) con acquisti mensili fissi.

Il capitolo è diviso in due parti: la prima è focalizzata sulle performance operando esclusivamente con Nasdaq-100, mentre la seconda riguarda i risultati del portafoglio diversificando gli investimenti con due ulteriori simboli (Dax-30, indice tedesco, e S\&P-500, altro indice americano).
Il programma, nel caso in cui vengano utilizzati più simboli, prevede la divisione equa del capitale iniziale tra i vari partecipanti.

%disclaimer
Il backtesting prende in considerazione fattori quali le commissioni sulle operazioni (0,05\% fisso per semplicità) e le tasse sul capital gain (26\% in Italia, applicate annualmente) per approssimare quanto meglio possibile la simulazione ad una situazione reale.


\section{Risultati investendo in Nasdaq-100}
%premessa su funzionamento PAC
Prima di procedere con i risultati è necessario fare una premessa.
In figura 4.1 viene mostrata l'evoluzione del portafoglio sotto la strategia PAC, che prevede acquisti mensili fissi indicativi di 100\$, in modo da dividere cosi il capitale iniziale in 50 operazioni eseguite in mesi consecutivi, per sfruttare il principio del dollar-cost averaging (in situazioni di incertezza, dividere gli acquisti in piccole operazioni distribuite nel tempo fornisce i migliori risultati).
La linea rossa rappresenta l'investimento cumulativo, che cresce linearmente nel tempo, mentre la linea blu indica il valore del portafoglio al netto delle spese (tasse e commissioni).
L'area ombreggiata in blu evidenzia il gain/loss netto, che rimane positivo per la maggior parte del periodo, raggiungendo un picco verso la fine del 2025.

%4.1 Timeline PAC, solo nasdaq
\begin{figure}[htbp]
    \centering
    \includegraphics[width=\textwidth]{Immagini/pac_investment_timeline_ndx.png}
    \caption{PAC - Timeline di acquisto}
    \label{fig:pac timeline}
\end{figure}

Dal grafico Monthly Purchase Points, i punti di acquisto mensili (cerchi blu) sono sovrapposti alla curva dei prezzi Nasdaq-100 (linea nera).
Si nota come questo metodo comporti che gli acquisti avvengano in fasi di mercato sia rialziste che ribassiste, sfruttando l'effetto di mediazione dei costi prima accennato (dollar-cost averaging).
Ad esempio, durante il calo del 2020 (legato alla pandemia COVID-19), gli acquisti a prezzi bassi contribuiscono a un recupero accelerato nel 2021-2022.
Tra il 2022-2024 l'indice passa in una fase di lateralizzazione, lasciando il capitale invariato.
Verso il 2025-2026, il portafoglio subisce una correzione, ma beneficia infine dell'ascesa dell'indice, riportando il gain in positivo.
E' importante notare la posizione dei dip di mercato: sovrapponendo questo grafico con quello visto precedentemente nel capitolo 2 (fig 2.2) riguardo i rendimenti del profilo 7 di MoneyFarm, si può osservare una similitudine importante negli andamenti, a sostegno della tesi iniziale che i R-A aggressivi investono principalmente in tech e che \textbf{il settore tech nel 2025 E' il Nasdaq-100}.


%4.2 equity totale, solo nasdaq
\begin{figure}[htbp]
    \centering
    \includegraphics[width=\textwidth]{Immagini/equity_comparison_all_profiles_ndx.png}
    \caption{NDX - Bilanci finali a confronto}
    \label{fig:equity nasdaq}
\end{figure}

%Grafico Equity solo Nasdaq
Il grafico 4.2 illustra l'andamento del capitale di bilancio per tutti i profili: alto rischio (rosso, HIGH), medio rischio (arancione, MEDIUM), basso rischio (verde, LOW), PAC (blu) e benchmark Moneyfarm (stessi colori e linee tratteggiate per i profili basso, medio e alto).

Le curve algoritmiche mostrano una crescita esponenziale, particolarmente pronunciata per il profilo alto rischio, che raggiunge valori superiori ai 50.000\$ durante il 2022, per poi stabilizzarsi a circa 37.000\$ nel 2026, partendo da un capitale iniziale di 5.000\$.
Si osserva un pattern a gradini nelle curve algoritmiche, dovuto alla strategia: la singola posizione legata al segnale (elaborato in backtesting.py) rimane aperta fino al raggiungimento del trailing stop; in quell'istante la posizione viene chiusa ed il gain/loss viene registrato.
Il profilo HIGH presenta i salti più ampi: sfrutta l'elevata volatilità dell'indice per guadagni importanti, ma sopporta drawdown significativi, specialmente nel 2023. 
Le curve di Moneyfarm sono rappresentate come linee continue a causa della mancanza di dati specifici riguardo le operazioni eseguite. Importante è il punto di "arrivo" che riflette i rendimenti dichiarati al netto di tasse e commissioni, calcolate dal programma prima della stampa finale.
Nel complesso le strategie si posizionano come previsto: HIGH presenta il maggior ritorno, seguito da MEDIUM e PAC, ed infine LOW.

%4.3 volatilità dei profili, solo nasdaq
\begin{figure}[htbp]
    \centering
    \includegraphics[width=\textwidth]{Immagini/volatility_comparison_all_profiles_ndx.png}
    \caption{NDX - Volatilità dei profili a confronto}
    \label{fig:volatilità nasdaq}
\end{figure}

Dall'istogramma in figura 4.3 si osserva come anche la volatilità sia stata distribuita correttamente fra i vari profili dell'algoritmo, segnale che i limiti fissati fossero congruenti ai livelli di rischio ricercati.
Le operazioni eseguite con profilo HIGH e MEDIUM (barre rossa e arancione) sono state esponenzialmente più volatili delle corrispettive del R-A di MoneyFarm.
Per il profilo LOW la volatilità risulta inferiore, ma si ricorda che lo stesso ha terminato i test in perdita.
Il profilo PAC risulta in linea con le aspettative: la volatilità segue strettamente i valori dell'indice.

%4.4 performance dei profili, solo nasdaq
\begin{figure}[htbp]
    \centering
    \includegraphics[width=\textwidth]{Immagini/performance_metrics_all_profiles_ndx.png}
    \caption{NDX - Performance dei profili a confronto}
    \label{fig:performance nasdaq}
\end{figure}

Il grafico in figura 4.4 presenta barre comparative per ritorno totale vs benchmark, Sharpe ratio, massimo drawdown ed un'ulteriore visualizzazione del ritorno dei profili.
Il capital gain dell'algoritmo HIGH supera il 2.067\%, contro l'88,5\% (netto) di Moneyfarm alto rischio.
Il profilo medio rischio algoritmico raggiunge il 148,5\%, mentre il basso termina in leggera perdita, -0,2\%.
Il profilo PAC risulta in gain a +115\%, un valore leggermente inferiore al MEDIUM.

Lo Sharpe Ratio, che misura il rendimento proporzionato al rischio, è positivo per i profili di trading algoritmici (0,58 alto, 0,33 medio, 0,53 basso) indicando un buon trade-off rischio-rendimento. Non brilla però se confrontato al benchmark di MoneyFarm (0.8 dichiarato) grazie alla bassa volatilità.
Il PAC si posiziona a -0,47 a causa appunto della elevata volatilità rispetto al guadagno potenziale.

Il Drawdown massimo (perdita massima sopportata) è più contenuto nei profili algoritmici rispetto al PAC, grazie all'utilizzo del trailing stop.

Nel complesso, questi risultati evidenziano la superiorità dell'algoritmo, specialmente per profili ad alto rischio, dove la volatilità dell'indice è sfruttata per rendimenti esponenziali, rispetto al R-A.
La strategia PAC genera un buon ritorno, nonostante l'elevata volatilità. Questa strategia si rivela robusta per investitori passivi, ma sotto-performante rispetto ai profili attivi in fasi di mercato rialzista.



\section{Risultati dopo la Diversificazione}
La seconda parte del capitolo analizza i risultati aggregati del portafoglio diversificato, affiancando gli investimenti in Nasdaq-100 a due ulteriori indici: Dax-30 ed S\&P-500.
Il programma prevede questa casualità, dividendo il capitale iniziale (mantenuto uguale a 5.000\$) equamente fra i simboli. Anche il PAC utilizza investimenti di 100/3\$ per simbolo, al mese.
Le successive figure 4.5 e 4.6 forniscono una vista olistica, mostrando come l'operazione di diversificazione mitighi i rischi rispetto al singolo simbolo.

%4.5 equity totale, diversificato
\begin{figure}[htbp]
    \centering
    \includegraphics[width=\textwidth]{Immagini/portfolio_cumulative_returns_log.png}
    \caption{Portfolio Diversificato - Bilanci finali a confronto}
    \label{fig:equity diversificato}
\end{figure}

Il grafico in figura 4.5 utilizza una scala logaritmica per il moltiplicatore del valore del portafoglio, facilitando il confronto di crescite esponenziali.
La linea rossa (HIGH algoritmico) domina, raggiungendo un moltiplicatore alla sesta verso il 2026, grazie ad operazioni aggressive su tutti i simboli. 
Le linee Moneyfarm (tratteggiate) rimangono uguali, riflettendo il fatto che il portafoglio è già diversificato anche durante i primi confronti.

La diversificazione riduce anche la volatilità: ad esempio, il profilo alto rischio mostra salti meno estremi rispetto al solo Nasdaq-100, beneficiando della correlazione bassa tra indici (il Dax-30 è più stabile durante la crisi). Si osserva lo stesso miglioramento anche sui profili MEDIUM e LOW.
Il PAC (blu) cresce in modo costante, superando i rendimenti MoneyFarm complessivi, ed addirittura anche il profilo MEDIUM.
Complessivamente, l'algoritmo alto rischio genera un ritorno del +157mila\%, con patrimonio finale netto di circa 7,5 milioni di dollari (dai 5.000\$ iniziali), dimostrando l'impatto della diversificazione.

%4.6 performance dei profili, diversificato
\begin{figure}[htbp]
    \centering
    \includegraphics[width=\textwidth]{Immagini/portfolio_performance_table.png}
    \caption{Portfolio Diversificato - Performance dei profili a confronto}
    \label{fig:performance diversificato}
\end{figure}

La tabella 4.6 riassume infine le metriche aggregate: il profilo HIGH si posiziona per primo, con però la maggior volatilità ed un drawdown medio.
In seconda posizione arriva il PAC, che dopo la diversificazione abbassa notevolmente la volatilità e quindi recupera in Sharpe Ratio. Il drawdown rimane comunque il peggiore.
In terza posizione, subito in coda al MEDIUM, si trova il profilo PAC; ed infine il profilo LOW, che chiude comunque in lieve guadagno (+16\%) battendo i benchmark di MoneyFarm (+1,8\%).

\chapter{Conclusioni e Sviluppi Futuri}

Questa tesi ha esaminato l’impatto delle architetture software sull’automazione finanziaria in Italia, confrontando un Robo-Advisor bancario (in questo caso MoneyFarm) con un algoritmo di trading semplice sviluppato ad-hoc (Moving-Average Crossover). 
L’ipotesi di ricerca, se un algoritmo personalizzato possa superare i rendimenti R-A a fronte di maggiore volatilità, è stata verificata tramite simulazioni su dati storici Nasdaq-100 (2018-2025) e successivamente su portafogli diversificati (Nasdaq-100, Dax-30 e S\&P-500).

\section{Verifica delle Ipotesi}
%Stabilità del RA vs Flessibilità Algo
Analizzando i risultati ottenuti, emerge chiaramente che non esiste un approccio universale: la scelta tra R-A e trading algoritmico dipende dal profilo dell’investitore, dal suo livello di istruzione e disponibilità d'impegno, e dalla tolleranza al rischio.
I R-A offrono rendimenti accettabili a livelli di rischio tollerabili, anche scegliendo i profili più azionari.
L’algoritmo, invece, raggiunge rendimenti sorprendentemente più alti nonostante la sua semplice implementazione, con volatilità molto elevata, evidenziando flessibilità per i periodi incerti ma anche rischi maggiori.
I dati di configurazione del software hanno rispettato le aspettative, posizionando nel corretto ordine i tre profili di rischio ipotizzati.
Anche il profilo PAC ha dato ottimi risultati, come visto in precedenza, battendo i ritorni dei R-A più aggressivi.

\section{Limiti Tecnici}
%Simulazione vs Real-time trading
L’analisi presenta limiti tecnici intrinseci alle simulazioni: il backtesting non cattura dinamiche reali, come slippage di esecuzione o eventi imprevedibili (come per esempio i flash crash), poichè basa i propri calcoli su dati storici tratti da Yahoo Finance, potenzialmente sovra-stimando i rendimenti ottenuti.
I costi di brokeraggio sono stati arrotondati per eccesso, ma ogni piattaforma propone commissioni diverse.
Inoltre, molto importante è ricordare che la strategia dell'algoritmo funziona sui dati del passato; questo non ne garantisce il funzionamento futuro.
Per quanto riguarda i Robo-Advisor, la ricerca si è basata su benchmark forniti dal distributore del servizio stesso, e non da organi ufficiali del settore.
Non va dimenticata infine la distinzione normativa: i R-A operano in regime amministrato, con la Banca che gestisce le tasse (26\% sul capital gain), semplificando la vita all’investitore.
Il trading algoritmico, invece, è tipicamente in regime dichiarativo, imponendo all’utente di dichiarare le plusvalenze autonomamente, un onere che, se mal gestito, può erodere i guadagni.
Questo dualismo rafforza l’attrattiva dei R-A per i neofiti, ma evidenzia i vantaggi per chi opta per soluzioni informate e personalizzate.

\section{Prospettive Future}
%Evoluzione AI e Regolamentazione al 2030
Guardando al futuro, è facile pensare che i servizi e le architetture software per investimenti si evolveranno naturalmente verso modelli AI ibridi, integrando robo-advisor passivi con algoritmi attivi per ottimizzare rendimenti e rischi.
Nel 2030, con l’EU AI Act che regola l'etica per lo sviluppo delle intelligenze artificiali, i robo-advisor italiani potrebbero incorporare AI generativa per consigli personalizzati, aumentando potenzialmente gli AUM oltre i \$200 miliardi.
Da non sottovalutare, inoltre, sarà il panorama open-source, che ad oggi conta già numerosi strumenti liberi per l’algo-trading, soprattutto nel campo delle Cryptovalute.
Un altro aspetto critico da considerare per il futuro sviluppo è la mancanza di scelta negli ambiti di investimento nei robo-advisor: l’utente è vincolato alle decisioni dell’istituzione senza poter selezionare settori specifici, delegando completamente la responsabilità.
Nel trading algoritmico, invece, l’utente assume un ruolo attivo, un elemento che, se da un lato aumenta il rischio emotivo, dall’altro favorisce l’apprendimento, la personalizzazione e di conseguenza l'investimento in settori che l'utente ritiene realmente importanti.


\section{Conclusioni}
Cosa conviene perciò all’investitore medio, spesso un giovane alle prime armi, “spaventato” dalla complessità del mondo della borsa? 
Per un principiante con bassa literacy finanziaria, un problema diffuso tra i giovani italiani e di tutto il mondo, un robo-advisor come MoneyFarm rappresenta la soluzione più pratica e sicura. 
Offre stabilità, diversificazione automatica e gestione fiscale in regime amministrato, con costi a livelli accessibili. 
Il cliente non deve fare altro che depositare i fondi e monitorare occasionalmente, ideale per chi cerca un passive income senza dedicare tempo allo studio dei mercati.

Anche la strategia del piano di accumulo del capitale (PAC) si rivela robusta per investitori passivi, soprattutto se il portafoglio viene correttamente diversificato.
Numerose Banche italiane, come Fineco, offrono servizi PAC che richiedono solamente la scelta di un profilo di investimento (profilazione tramite questioniario MiFID) per generare consigli d'investimento ad-hoc. Anche questi sono servizi in regime amministrato "senza pensieri".

%Conclusione umana
I migliori risultati, però, con ritorni esponenzialmente più elevati della concorrenza, rimangono nell'ambito del trading algoritmico.
Quindi sorge spontaneamente la domanda: se tutti avessero a disposizione un sistema automatico per generare profitto, cosa rappresenterebbero gli investimenti?
In un mondo di automazione totale, il mercato rischierebbe di ridursi a pura speculazione, con prezzi dettati da algoritmi piuttosto che da valutazioni fondamentali, come per esempio l'impatto di una determinata azienda quotata sull'innovazione del suo settore, o l'importanza di un'altra per ridurre l'impronta dell'uomo sull'ambiente.
Come visto nel capitolo 3, il 2025 appena conclusosi è stato simbolo proprio di questo: il mercato è stato dominato dai bilanci riportati dalle principali aziende che si occupano di AI, a cui seguono finanziamenti anche "involontari" ed un ciclo vizioso d'investimento l'una nell'altra, che ha palesemente generato una bolla.
L’AI può ottimizzare, ma l’elemento umano (curiosità, etica e responsabilità) resta essenziale per un ecosistema finanziario sostenibile.
Non preoccuparsi di come e dove vengano posti i propri risparmi, e lasciare alle macchine il compito della gestione patrimoniale, sarebbe la ricetta per un disastro.

Insomma, per concludere, la digitalizzazione dei processi ha permesso la nascita di numerose opzioni "low maintenance", quindi l'approccio al mondo della borsa non è mai stato cosi semplice.
L'unico ostacolo da superare è l'istruzione: per i giovani, il vero passive income inizia dall'educazione finanziaria, che trasforma la paura in opportunità.



\chapter*{Ringraziamenti}
\addcontentsline{toc}{chapter}{Ringraziamenti}

Prima di tutto un doveroso ringraziamento alla mia famiglia, mamma Monica, papà Giambattista e Riccardo, i miei nonni Luigi e Orsolina, per avermi sempre incoraggiato e sostenuto durante il percorso.

Ringrazio la mia compagna Michela, che mi ha conosciuto quando il percorso era appena iniziato, e sapeva in primis a che tipo di sacrifici andava incontro volendosi affiancare ad una persona con un lavoro a tempo pieno unito alla volontà di portare a termine gli studi. Questo è stato il vero investimento.

Un ringraziamento a Davide, che mi ha pazientemente saputo insegnare che più del punto di arrivo, nella vita conta il senso che si dà ai propri passi.

Un ringraziamento al mio relatore, Prof. Bellone, per la disponibilità dimostrata da subito e per aver sempre accolto le mie idee senza obiezioni, ma anzi con nuovi spunti di riflessione.

Infine un ringraziamento speciale a tutte le persone a me care che hanno saputo convincermi a riprovarci. 

Per me terminare questo percorso significa più di un semplice Titolo, ma il sapere che nella vita non è finita finchè non decidi d'esser stato sconfitto.

\nocite{*}
\printbibliography

\end{document}

%---------------------------END-------------------------------------