\chapter{Conclusioni e Sviluppi Futuri}

Questa tesi ha esaminato l’impatto delle architetture software sull’automazione finanziaria in Italia, confrontando un Robo-Advisor bancario (in questo caso MoneyFarm) con un algoritmo di trading semplice sviluppato ad-hoc (Moving-Average Crossover). 
L’ipotesi di ricerca, se un algoritmo personalizzato possa superare i rendimenti R-A a fronte di maggiore volatilità, è stata verificata tramite simulazioni su dati storici Nasdaq-100 (2018-2025) e successivamente su portafogli diversificati (Nasdaq-100, Dax-30 e S\&P-500).

\section{Verifica delle Ipotesi}
%Stabilità del RA vs Flessibilità Algo
Analizzando i risultati ottenuti, emerge chiaramente che non esiste un approccio universale: la scelta tra R-A e trading algoritmico dipende dal profilo dell’investitore, dal suo livello di istruzione e disponibilità d'impegno, e dalla tolleranza al rischio.
I R-A offrono rendimenti accettabili a livelli di rischio tollerabili, anche scegliendo i profili più azionari.
L’algoritmo, invece, raggiunge rendimenti sorprendentemente più alti nonostante la sua semplice implementazione, con volatilità molto elevata, evidenziando flessibilità per i periodi incerti ma anche rischi maggiori.
I dati di configurazione del software hanno rispettato le aspettative, posizionando nel corretto ordine i tre profili di rischio ipotizzati.
Anche il profilo PAC ha dato ottimi risultati, come visto in precedenza, battendo i ritorni dei R-A più aggressivi.

\section{Limiti Tecnici}
%Simulazione vs Real-time trading
L’analisi presenta limiti tecnici intrinseci alle simulazioni: il backtesting non cattura dinamiche reali, come slippage di esecuzione o eventi imprevedibili (come per esempio i flash crash), poichè basa i propri calcoli su dati storici tratti da Yahoo Finance, potenzialmente sovra-stimando i rendimenti ottenuti.
I costi di brokeraggio sono stati arrotondati per eccesso, ma ogni piattaforma propone commissioni diverse.
Inoltre, molto importante è ricordare che la strategia dell'algoritmo funziona sui dati del passato; questo non ne garantisce il funzionamento futuro.
Per quanto riguarda i Robo-Advisor, la ricerca si è basata su benchmark forniti dal distributore del servizio stesso, e non da organi ufficiali del settore.
Non va dimenticata infine la distinzione normativa: i R-A operano in regime amministrato, con la Banca che gestisce le tasse (26\% sul capital gain), semplificando la vita all’investitore.
Il trading algoritmico, invece, è tipicamente in regime dichiarativo, imponendo all’utente di dichiarare le plusvalenze autonomamente, un onere che, se mal gestito, può erodere i guadagni.
Questo dualismo rafforza l’attrattiva dei R-A per i neofiti, ma evidenzia i vantaggi per chi opta per soluzioni informate e personalizzate.

\section{Prospettive Future}
%Evoluzione AI e Regolamentazione al 2030
Guardando al futuro, è facile pensare che i servizi e le architetture software per investimenti si evolveranno naturalmente verso modelli AI ibridi, integrando robo-advisor passivi con algoritmi attivi per ottimizzare rendimenti e rischi.
Nel 2030, con l’EU AI Act che regola l'etica per lo sviluppo delle intelligenze artificiali, i robo-advisor italiani potrebbero incorporare AI generativa per consigli personalizzati, aumentando potenzialmente gli AUM oltre i \$200 miliardi.
Da non sottovalutare, inoltre, sarà il panorama open-source, che ad oggi conta già numerosi strumenti liberi per l’algo-trading, soprattutto nel campo delle Cryptovalute.
Un altro aspetto critico da considerare per il futuro sviluppo è la mancanza di scelta negli ambiti di investimento nei robo-advisor: l’utente è vincolato alle decisioni dell’istituzione senza poter selezionare settori specifici, delegando completamente la responsabilità.
Nel trading algoritmico, invece, l’utente assume un ruolo attivo, un elemento che, se da un lato aumenta il rischio emotivo, dall’altro favorisce l’apprendimento, la personalizzazione e di conseguenza l'investimento in settori che l'utente ritiene realmente importanti.


\section{Conclusioni}
Cosa conviene perciò all’investitore medio, spesso un giovane alle prime armi, “spaventato” dalla complessità del mondo della borsa? 
Per un principiante con bassa literacy finanziaria, un problema diffuso tra i giovani italiani e di tutto il mondo, un robo-advisor come MoneyFarm rappresenta la soluzione più pratica e sicura. 
Offre stabilità, diversificazione automatica e gestione fiscale in regime amministrato, con costi a livelli accessibili. 
Il cliente non deve fare altro che depositare i fondi e monitorare occasionalmente, ideale per chi cerca un passive income senza dedicare tempo allo studio dei mercati.

Anche la strategia del piano di accumulo del capitale (PAC) si rivela robusta per investitori passivi, soprattutto se il portafoglio viene correttamente diversificato.
Numerose Banche italiane, come Fineco, offrono servizi PAC che richiedono solamente la scelta di un profilo di investimento (profilazione tramite questioniario MiFID) per generare consigli d'investimento ad-hoc. Anche questi sono servizi in regime amministrato "senza pensieri".

%Conclusione umana
I migliori risultati, però, con ritorni esponenzialmente più elevati della concorrenza, rimangono nell'ambito del trading algoritmico.
Quindi sorge spontaneamente la domanda: se tutti avessero a disposizione un sistema automatico per generare profitto, cosa rappresenterebbero gli investimenti?
In un mondo di automazione totale, il mercato rischierebbe di ridursi a pura speculazione, con prezzi dettati da algoritmi piuttosto che da valutazioni fondamentali, come per esempio l'impatto di una determinata azienda quotata sull'innovazione del suo settore, o l'importanza di un'altra per ridurre l'impronta dell'uomo sull'ambiente.
Come visto nel capitolo 3, il 2025 appena conclusosi è stato simbolo proprio di questo: il mercato è stato dominato dai bilanci riportati dalle principali aziende che si occupano di AI, a cui seguono finanziamenti anche "involontari" ed un ciclo vizioso d'investimento l'una nell'altra, che ha palesemente generato una bolla.
L’AI può ottimizzare, ma l’elemento umano (curiosità, etica e responsabilità) resta essenziale per un ecosistema finanziario sostenibile.
Non preoccuparsi di come e dove vengano posti i propri risparmi, e lasciare alle macchine il compito della gestione patrimoniale, sarebbe la ricetta per un disastro.

Insomma, per concludere, la digitalizzazione dei processi ha permesso la nascita di numerose opzioni "low maintenance", quindi l'approccio al mondo della borsa non è mai stato cosi semplice.
L'unico ostacolo da superare è l'istruzione: per i giovani, il vero passive income inizia dall'educazione finanziaria, che trasforma la paura in opportunità.

