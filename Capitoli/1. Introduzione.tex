\chapter{Introduzione}
\section{Contesto}

Per rispondere a questa domanda, è prima necessario comprendere il contesto storico.

La crisi finanziaria associata alla pandemia da COVID-19 ha rappresentato un catalizzatore significativo per l'evoluzione della cultura finanziaria tra i giovani di tutto il mondo, evidenziando vulnerabilità pre-esistenti e stimolando una maggiore consapevolezza: inflazione alle stelle, stipendi congelati e cultura sociale del paragone fanno sorgere nuove necessità, seppure i dati indichino l'assenza di un incremento netto dei livelli di alfabetizzazione finanziaria - punteggio italiano OCSE PISA di 484 nel 2022, al di sotto della media OCSE (498), con il 18\% dei 15enni sotto il livello base.

In Italia, quasi il 40\% dei giovani (18-34 anni) è incapace di fronteggiare imprevisti di 2.000€. Inoltre, il 54\% riferisce di percepire la crisi come una minaccia grave. Tutto ciò accentua le disuguaglianze di genere e territoriali (ad esempio donne e residenti al Sud).

Questa diffusa paura legata alla difficoltà intrinseca di capire il complesso mondo della borsa genera un mix efficace: il mondo dei social media viene letteralmente invaso dalle figure dei "guru finanziari", le piattaforme di trading spopolano sugli app-store e l'ideologia generale tende alla ricerca della cosiddetta work-life balance e del passive income.

Anche le Banche presentano la loro proposta per rispondere ad un pubblico digitalizzato, con pochi fondi di partenza e che non può quindi sostenere commissioni alte: nascono i Robo-Advisor, strumenti automatizzati per la gestione degli investimenti. 
Il cliente retail può iniziare con cifre modeste (solitamente da 500€ a 5.000€), sostenendo commissioni basse (in media fra 0.25-0.5\%), a cui associa un profilo di rischio individuato tramite questionario, e a cui può eventualmente aggiungere mensilmente una cifra.

\section{Ipotesi}

Questa tesi vuole analizzare l’impatto di questi servizi, confrontando i rendimenti di un R-A in regime amministrato (verrà preso in esame in particolare MoneyFarm, uno fra i più diffusi e redditizi, e soprattutto che rende disponibili pubblicamente i propri dati) con un algoritmo di trading “semplice” sviluppato ad hoc (in questo caso verrà utilizzata la strategia moving-average crossover) che opera in loop su una piattaforma gestita da broker.

L’ipotesi di ricerca valuta se, ad oggi, sia possibile creare ed utilizzare un algoritmo personalizzato che possa imitare il metodo d’investimento di un R-A, superarne i rendimenti a fronte di un maggiore involvement da parte dell’utente, con un livello di rischio simile ai profili forniti dalle Banche.
Questo per tutelare l’utente con literacy bassa ed insieme fornire la tranquillità necessaria rimuovendo la componente dell’emotività umana dalla strategia.
Quanto può influire, se non meglio costare, un minimo livello di cultura finanziaria, sorprendentemente assente fra i giovani italiani, sui risparmi?


\section{Metodologia}

La metodologia prevede la creazione di un algoritmo per l’operatività, che raccoglie dati da fonti ufficiali e li utilizza per implementare una specifica strategia (a valori configurabili).
Verranno effettuate simulazioni empiriche tramite Python su dataset storici, considerando fattori reali quali commissioni e tasse italiane. 
I risultati verranno confrontati con le performance dichiarate da MoneyFarm, valutando metriche come rendimento (patrimonio finale), volatilità (a simboleggiare il rischio), drawdown (massima perdita), ed indice di Sharpe (indice rischio/rendimento).

La tesi presenta il diagramma UML dell’architettura software utilizzata, correlato a specifiche descrizioni dei vari blocchi necessari all'espletamento delle funzioni.

Il lavoro contribuisce al dibattito sull'automazione finanziaria, offrendo un'analisi tecnica per investitori e sviluppatori.

È importante evidenziare che, attraverso i materiali qui presenti, NON vengono forniti consigli finanziari né consulenza finanziaria. Le opinioni tecniche condivise hanno il solo scopo di formazione ed informazione. La decisione finale riguardante ogni investimento è una scelta personale e individuale del lettore.
