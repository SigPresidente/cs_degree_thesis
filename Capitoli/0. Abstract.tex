\chapter*{Abstract} 
\addcontentsline{toc}{chapter}{Abstract}

L'automazione finanziaria, potenziata e supportata dalla diffusione delle intelligenze artificiali (AI) degli ultimi anni, sta ridefinendo il panorama degli investimenti retail in Italia e a livello globale.

Nel 2025 il mercato dei robo-advisor (R-A) ha raggiunto un valore stimato di \$14 miliardi, con una crescita annua (dati CAGR) del 30\%, trainata da piattaforme bancarie come MoneyFarm, che offrono soluzioni accessibili con commissioni basse (0.25-0.5\%).

Parallelamente, il trading online, ed in modo più specifico il trading con algoritmi reso popolare da piattaforme come MetaTrader5, consente agli investitori privati di sviluppare strategie personalizzate, con vari gradi di rischio e potenziale rendimento.

In Italia il contesto normativo (MiFID II) distingue il regime amministrato con responsabilità fiscale gestita dalla banca, dal regime dichiarativo, che invece richiede competenze tecniche e responsabilità diretta.

Questo dualismo solleva una domanda cruciale: quale approccio è più adatto all’investitore medio di oggi?