\chapter{Risultati e Confronto}

In questo capitolo vengono analizzati i risultati ottenuti tramite backtesting della strategia sui dati storici forniti da Yahoo Finance.
La simulazione confronta le performance della strategia algoritmica creata ad-hoc con i valori di performance indicati sul sito web ufficiale di Moneyfarm, nel periodo dal 2018 ad oggi.
I grafici contengono i risultati per i tre diversi profili di rischio (basso, medio e alto), oltre a una strategia di investimento periodico (PAC) con acquisti mensili fissi.

Il capitolo è diviso in due parti: la prima è focalizzata sulle performance operando esclusivamente con Nasdaq-100, mentre la seconda riguarda i risultati del portafoglio diversificando gli investimenti con due ulteriori simboli (Dax-30, indice tedesco, e S\&P-500, altro indice americano).
Il programma, nel caso in cui vengano utilizzati più simboli, prevede la divisione equa del capitale iniziale tra i vari partecipanti.

%disclaimer
Il backtesting prende in considerazione fattori quali le commissioni sulle operazioni (0,05\% fisso per semplicità) e le tasse sul capital gain (26\% in Italia, applicate annualmente) per approssimare quanto meglio possibile la simulazione ad una situazione reale.


\section{Risultati investendo in Nasdaq-100}
%premessa su funzionamento PAC
Prima di procedere con i risultati è necessario fare una premessa.
In figura 4.1 viene mostrata l'evoluzione del portafoglio sotto la strategia PAC, che prevede acquisti mensili fissi indicativi di 100\$, in modo da dividere cosi il capitale iniziale in 50 operazioni eseguite in mesi consecutivi, per sfruttare il principio del dollar-cost averaging (in situazioni di incertezza, dividere gli acquisti in piccole operazioni distribuite nel tempo fornisce i migliori risultati).
La linea rossa rappresenta l'investimento cumulativo, che cresce linearmente nel tempo, mentre la linea blu indica il valore del portafoglio al netto delle spese (tasse e commissioni).
L'area ombreggiata in blu evidenzia il gain/loss netto, che rimane positivo per la maggior parte del periodo, raggiungendo un picco verso la fine del 2025.

%4.1 Timeline PAC, solo nasdaq
\begin{figure}[htbp]
    \centering
    \includegraphics[width=\textwidth]{Immagini/pac_investment_timeline_ndx.png}
    \caption{PAC - Timeline di acquisto}
    \label{fig:pac timeline}
\end{figure}

Dal grafico Monthly Purchase Points, i punti di acquisto mensili (cerchi blu) sono sovrapposti alla curva dei prezzi Nasdaq-100 (linea nera).
Si nota come questo metodo comporti che gli acquisti avvengano in fasi di mercato sia rialziste che ribassiste, sfruttando l'effetto di mediazione dei costi prima accennato (dollar-cost averaging).
Ad esempio, durante il calo del 2020 (legato alla pandemia COVID-19), gli acquisti a prezzi bassi contribuiscono a un recupero accelerato nel 2021-2022.
Tra il 2022-2024 l'indice passa in una fase di lateralizzazione, lasciando il capitale invariato.
Verso il 2025-2026, il portafoglio subisce una correzione, ma beneficia infine dell'ascesa dell'indice, riportando il gain in positivo.
E' importante notare la posizione dei dip di mercato: sovrapponendo questo grafico con quello visto precedentemente nel capitolo 2 (fig 2.2) riguardo i rendimenti del profilo 7 di MoneyFarm, si può osservare una similitudine importante negli andamenti, a sostegno della tesi iniziale che i R-A aggressivi investono principalmente in tech e che \textbf{il settore tech nel 2025 E' il Nasdaq-100}.


%4.2 equity totale, solo nasdaq
\begin{figure}[htbp]
    \centering
    \includegraphics[width=\textwidth]{Immagini/equity_comparison_all_profiles_ndx.png}
    \caption{NDX - Bilanci finali a confronto}
    \label{fig:equity nasdaq}
\end{figure}

%Grafico Equity solo Nasdaq
Il grafico 4.2 illustra l'andamento del capitale di bilancio per tutti i profili: alto rischio (rosso, HIGH), medio rischio (arancione, MEDIUM), basso rischio (verde, LOW), PAC (blu) e benchmark Moneyfarm (stessi colori e linee tratteggiate per i profili basso, medio e alto).

Le curve algoritmiche mostrano una crescita esponenziale, particolarmente pronunciata per il profilo alto rischio, che raggiunge valori superiori ai 50.000\$ durante il 2022, per poi stabilizzarsi a circa 37.000\$ nel 2026, partendo da un capitale iniziale di 5.000\$.
Si osserva un pattern a gradini nelle curve algoritmiche, dovuto alla strategia: la singola posizione legata al segnale (elaborato in backtesting.py) rimane aperta fino al raggiungimento del trailing stop; in quell'istante la posizione viene chiusa ed il gain/loss viene registrato.
Il profilo HIGH presenta i salti più ampi: sfrutta l'elevata volatilità dell'indice per guadagni importanti, ma sopporta drawdown significativi, specialmente nel 2023. 
Le curve di Moneyfarm sono rappresentate come linee continue a causa della mancanza di dati specifici riguardo le operazioni eseguite. Importante è il punto di "arrivo" che riflette i rendimenti dichiarati al netto di tasse e commissioni, calcolate dal programma prima della stampa finale.
Nel complesso le strategie si posizionano come previsto: HIGH presenta il maggior ritorno, seguito da MEDIUM e PAC, ed infine LOW.

%4.3 volatilità dei profili, solo nasdaq
\begin{figure}[htbp]
    \centering
    \includegraphics[width=\textwidth]{Immagini/volatility_comparison_all_profiles_ndx.png}
    \caption{NDX - Volatilità dei profili a confronto}
    \label{fig:volatilità nasdaq}
\end{figure}

Dall'istogramma in figura 4.3 si osserva come anche la volatilità sia stata distribuita correttamente fra i vari profili dell'algoritmo, segnale che i limiti fissati fossero congruenti ai livelli di rischio ricercati.
Le operazioni eseguite con profilo HIGH e MEDIUM (barre rossa e arancione) sono state esponenzialmente più volatili delle corrispettive del R-A di MoneyFarm.
Per il profilo LOW la volatilità risulta inferiore, ma si ricorda che lo stesso ha terminato i test in perdita.
Il profilo PAC risulta in linea con le aspettative: la volatilità segue strettamente i valori dell'indice.

%4.4 performance dei profili, solo nasdaq
\begin{figure}[htbp]
    \centering
    \includegraphics[width=\textwidth]{Immagini/performance_metrics_all_profiles_ndx.png}
    \caption{NDX - Performance dei profili a confronto}
    \label{fig:performance nasdaq}
\end{figure}

Il grafico in figura 4.4 presenta barre comparative per ritorno totale vs benchmark, Sharpe ratio, massimo drawdown ed un'ulteriore visualizzazione del ritorno dei profili.
Il capital gain dell'algoritmo HIGH supera il 2.067\%, contro l'88,5\% (netto) di Moneyfarm alto rischio.
Il profilo medio rischio algoritmico raggiunge il 148,5\%, mentre il basso termina in leggera perdita, -0,2\%.
Il profilo PAC risulta in gain a +115\%, un valore leggermente inferiore al MEDIUM.

Lo Sharpe Ratio, che misura il rendimento proporzionato al rischio, è positivo per i profili di trading algoritmici (0,58 alto, 0,33 medio, 0,53 basso) indicando un buon trade-off rischio-rendimento. Non brilla però se confrontato al benchmark di MoneyFarm (0.8 dichiarato) grazie alla bassa volatilità.
Il PAC si posiziona a -0,47 a causa appunto della elevata volatilità rispetto al guadagno potenziale.

Il Drawdown massimo (perdita massima sopportata) è più contenuto nei profili algoritmici rispetto al PAC, grazie all'utilizzo del trailing stop.

Nel complesso, questi risultati evidenziano la superiorità dell'algoritmo, specialmente per profili ad alto rischio, dove la volatilità dell'indice è sfruttata per rendimenti esponenziali, rispetto al R-A.
La strategia PAC genera un buon ritorno, nonostante l'elevata volatilità. Questa strategia si rivela robusta per investitori passivi, ma sotto-performante rispetto ai profili attivi in fasi di mercato rialzista.



\section{Risultati dopo la Diversificazione}
La seconda parte del capitolo analizza i risultati aggregati del portafoglio diversificato, affiancando gli investimenti in Nasdaq-100 a due ulteriori indici: Dax-30 ed S\&P-500.
Il programma prevede questa casualità, dividendo il capitale iniziale (mantenuto uguale a 5.000\$) equamente fra i simboli. Anche il PAC utilizza investimenti di 100/3\$ per simbolo, al mese.
Le successive figure 4.5 e 4.6 forniscono una vista olistica, mostrando come l'operazione di diversificazione mitighi i rischi rispetto al singolo simbolo.

%4.5 equity totale, diversificato
\begin{figure}[htbp]
    \centering
    \includegraphics[width=\textwidth]{Immagini/portfolio_cumulative_returns_log.png}
    \caption{Portfolio Diversificato - Bilanci finali a confronto}
    \label{fig:equity diversificato}
\end{figure}

Il grafico in figura 4.5 utilizza una scala logaritmica per il moltiplicatore del valore del portafoglio, facilitando il confronto di crescite esponenziali.
La linea rossa (HIGH algoritmico) domina, raggiungendo un moltiplicatore alla sesta verso il 2026, grazie ad operazioni aggressive su tutti i simboli. 
Le linee Moneyfarm (tratteggiate) rimangono uguali, riflettendo il fatto che il portafoglio è già diversificato anche durante i primi confronti.

La diversificazione riduce anche la volatilità: ad esempio, il profilo alto rischio mostra salti meno estremi rispetto al solo Nasdaq-100, beneficiando della correlazione bassa tra indici (il Dax-30 è più stabile durante la crisi). Si osserva lo stesso miglioramento anche sui profili MEDIUM e LOW.
Il PAC (blu) cresce in modo costante, superando i rendimenti MoneyFarm complessivi, ed addirittura anche il profilo MEDIUM.
Complessivamente, l'algoritmo alto rischio genera un ritorno del +157mila\%, con patrimonio finale netto di circa 7,5 milioni di dollari (dai 5.000\$ iniziali), dimostrando l'impatto della diversificazione.

%4.6 performance dei profili, diversificato
\begin{figure}[htbp]
    \centering
    \includegraphics[width=\textwidth]{Immagini/portfolio_performance_table.png}
    \caption{Portfolio Diversificato - Performance dei profili a confronto}
    \label{fig:performance diversificato}
\end{figure}

La tabella 4.6 riassume infine le metriche aggregate: il profilo HIGH si posiziona per primo, con però la maggior volatilità ed un drawdown medio.
In seconda posizione arriva il PAC, che dopo la diversificazione abbassa notevolmente la volatilità e quindi recupera in Sharpe Ratio. Il drawdown rimane comunque il peggiore.
In terza posizione, subito in coda al MEDIUM, si trova il profilo PAC; ed infine il profilo LOW, che chiude comunque in lieve guadagno (+16\%) battendo i benchmark di MoneyFarm (+1,8\%).